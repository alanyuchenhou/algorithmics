\documentclass{article}
\usepackage{listings}
\usepackage{amsmath}
\usepackage{fullpage}
\usepackage{tabularx}
\usepackage{graphicx}
\usepackage{cite}
\begin{document}
\lstset{language=python, tabsize=4}
\title{CS 516 final}
\author{Yuchen Hou}
\maketitle

\section{decidability}

\subsection{A-TT is undecidable}
A TT is a machine with one variable x and instruction set:
\begin{enumerate}
	\item x := 2x
	\item x := 3x
	\item x := x/2
	\item x := x/3
	\item x mod 2 = 0
	\item x mod 3 = 0
\end{enumerate}

\subsubsection{problem}
\begin{enumerate}
	\item given: TT T
	\item question: does T halt?
\end{enumerate}

\subsubsection{proof}
Proof by reduction (from A-2CM to A-TT) and contradiction (assumption: A-TT is 
decidable).
From the assumption, there is decider M (input = [N], where N is a TT) to 
decide A-TT.
We can construct decider O (input = [P], where P is a 2CM) to decide A-2CM:
\begin{enumerate}
	\item construct a TT Q to halt if P halts:
	\begin{enumerate}
		\item simulate P x++ with x := 2x
		\item simulate P y++ with x := 3x
		\item ...
	\end{enumerate}
	\item if Q halts (decidable by M): accept
	\item else: reject
\end{enumerate}
This is a contradiction with A-2CM is undecidable. Therefore A-TT is 
undecidable.

\subsection{L is semi-linear}
\begin{enumerate}
	\item L is Reg
	\item L is CFL
	\item L = $ semi-linear \cap Reg $
	\item P is Presburger formula and P defines \#(L)
\end{enumerate}

\subsection{E-good-paths is decidable}
Each edge has a color.
A walk is the sequence of edge colors.
\subsubsection{problem: E-good-paths}
\begin{enumerate}
	\item given: directed graph G(V, E), v1, v2 in V
	\item question: is there a path(v1, v2) w with constraint: $ \#red(w) = 
	\#orange(w) = \#yellow(w) $
\end{enumerate}

\subsubsection{proof}
Proof by reduction (from E-good-paths to E-semi-linear).
There is a decider N (input = [L], where L is Semi-linear) to decide 
E-semi-linear.
We can construct decider M (input = [G, v1, v2], where G is a graph, v1, v2 are 
nodes in the graph) to decide E-good-graph:
\begin{enumerate}
	\item $ L_1 = \{w: w \in paths(v1, v2)\} \in Reg $
	\item $ L_2 = \{w: \#red(w) = \#orange(w) = \#yellow(w) \} \in Semi-linear $
	\item L = $ L1 \cap L2 \in Semi-linear $
	\item if $ L = \emptyset $ (decidable by N): reject
	\item else: accept
\end{enumerate}
Therefore E-good-paths is decidable.

\subsection{E-good-instruction-sequences is decidable}
there is a water pump, pumping water from river 1 and river 2 to a pool.
Pump instruction set (instruction[i] = [river[1], river[2]]; river = [min, 
max]):
\begin{enumerate}
	\item red = [3,4], [2, 10]
	\item blue = [2, 11], [0,1]
	\item ...
	\item purple = stop	
\end{enumerate}

\subsubsection{problem: E-good-instruction-sequences}
\begin{enumerate}
	\item given an instruction set
	\item question: is there a instruction sequence w satisfying constraint 
	river1 > 2 river2
\end{enumerate}

\subsubsection{proof}
Proof by reduction (from E-good-instruction-sequences to 
satisfiability-mixed-linear-formula).
There is decider M (input = [L], where L is a mixed-linear language) to decide 
E-mixed-linear.
Construct decider N (input = [instruction set]) to decide 
E-good-instruction-sequences:
\begin{enumerate}
	\item construct the language of acceptable sequences:
	\begin{align*}
	L = \{w: & river1 > 3\#red(w) + 2\#blue(w) + ... \\
	& river1 < 4\#red(w) + 11\#blue(w) + ... \\
	& river2 > 2\#red(w) + 0\#blue(w) + ... \\
	& river2 < 10\#red(w) + 1\#blue(w) + ... \\
	& river1 > 2river2\} \in mixed-linear
	\end{align*}
	\item if $ L = \emptyset $ (decidable by M): reject
	\item else: accept
\end{enumerate}
Therefore E-good-instruction-sequences is decidable.

\section{P VS NP}
\subsection{definitions}
P, NP, NP-complete, reduction, PSPACE, co-NP

\subsection{example: equal-boolean is in P with false assumption}
false assumption: SAT is in P

\subsubsection{problem: equal-boolean}
\begin{enumerate}
	\item given: boolean formula A, B
	\item question: $ \forall X A(X) = B(X) $?
\end{enumerate}

\subsubsection{proof: algorithm design}
Proof by reduction (from equal-boolean to SAT).
$ \exists $ deterministic polynomial-time decider M (input = [A], where A is a 
boolean formula) to decide SAT. We can construct polynomial-time decider N 
(input = [A, B], where A and B are boolean formulas) to decide boolean equality:
\begin{enumerate}
	\item reduce [A, B] to C: $ A \oplus B $
	\item if C is satisfiable (decidable by M): reject
	\item else: accept
\end{enumerate}
Therefore equal-boolean is in P.

\subsection{example: HC1 is in NP}

\subsubsection{problem: HC1}
\begin{enumerate}
	\item given: direct graph G
	\item question: is there a walk w on G, s.t. w passes every node in G
\end{enumerate}

\subsubsection{proof: algorithm design}
construct deterministic polynomial-time decider (input = [G], where G is a 
directed graph) to decide HC1:
\begin{enumerate}
	\item DAG = SCC(G)
	\item if nodes in DAG are connected in one sequence: accept
	\item else: reject
\end{enumerate}
Therefore HC1 is in P and of course in NP.

\subsection{example: factorization is in NP}

\subsubsection{problem: factorization}
\begin{enumerate}
	\item given: number n
	\item question: are there p, q s.t. p, q are primes and pq = n?
\end{enumerate}

\subsubsection{proof: algorithm design}
construct non-deterministic polynomial-time decider (input = [n], where n is a 
number) to decide factorization:
\begin{enumerate}
	\item guess p, q // $ |q| < |n|, |q| < |n| $
	\item if n = pq and p, q are primes: accept // deterministic polynomial-time
	\item else: reject
\end{enumerate}
Therefore factorization is in NP.

\subsection{X is in NP-complete}
\begin{enumerate}
	\item $ X \in NP $
	\item $ Y \in NP-complete \land Y \leq_m X $
\end{enumerate}
Therefore X is in NP-complete.

\end{document}