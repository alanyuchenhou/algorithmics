\documentclass{article}
\usepackage{listings}
\usepackage{amsmath}
\usepackage{fullpage}
\usepackage{tabularx}
\usepackage{graphicx}
\usepackage{cite}
\begin{document}
\lstset{language=python, tabsize=4}
\title{CS 516 semilinear}
\author{Yuchen Hou}
\maketitle

\section{P VS NP}
\subsection{definitions}
P, NP, NP-complete, reduction, PSPACE, co-NP

\subsection{example: factorization is in NP}

\subsubsection{problem}
given: number n
question: are there p, q in prime, s.t. n = pq

\subsubsection{proof: algorithm design}

\begin{lstlisting}
	def algorithm(n):
		guess p, q // |q| < |n|, |q| < |n|
		if n = pq and q is prime and q is prime: // deterministic polynomial time
			return yes
		else:
			return no
\end{lstlisting}

\subsection{X is in NP-complete}

\subsubsection{proof}
\begin{lstlisting}
	show X is in NP
	find Y in NP-complete
	reduce Y to X
\end{lstlisting}

\subsection{example: HC1 is NP}

\subsubsection{problem}
given: direct graph G
question: is there a walk w on G, s.t. w passes every node in G

\subsubsection{proof: algorithm design}
prove HC1 is in P by providing a algorithm
\begin{lstlisting}
	def algorithm(G):
		DAG = SCC(G)
		assert(DAG is connected in one sequence)
\end{lstlisting}

\subsection{example}
suppose SAT in P, design polynomial algorithm to solve:

\subsubsection{problem}
given boolean formula A,B
question: A = B?

\subsubsection{solution: algorithm design}
\begin{itemize}
	\item there is polynomial time algorithm A to solve SAT
	\item assert (A xor B) is not satisfiable
\end{itemize}

\section{emptiness proof}

\subsection{emptiness is decidable}
map the problem to a language = all words intersect with constraints
show the language is semi-linear (L is reg or L is CF)

\subsubsection{emptiness is undecidable}
2CM

\section{halting is decidable}

\section{E-L is decidable}
\begin{enumerate}
	\item show L is semi-linear
	\item E-L is decidable
\end{enumerate}

\section{F(x1, x2 ... xk) is Presburger}
\begin{enumerate}
	\item x1...xk don't move, then directly write F into a Presburger formular. e.g. homework 2
	\item construct L such that L is semiliear and \#(L) is definable by F. e.g. homework 5
\end{enumerate}

\section{L is semi-linear}
\begin{itemize}
	\item show L is regular
	\item show L is CFL
	\item show L = semi-linear intersect regular intersect semi-linear-commutative
\end{itemize}

\section{graph example}

\subsection{problem}
\begin{itemize}
	\item given: directed graph G, each edge has a color; 2 nodes v1, v2 in G
	\item on a walk from v1 to v2, we count the edges of different colors
	\item G is good if there is walk that meets condition: red count = orange count = yellow count
	\item question is G good?
\end{itemize}

\subsection{problem is decidable}
\begin{itemize}
	\item L = \{walk: walk meets condition\}
	\item L = L1 intersect L2
	\item L1 = \{all walks from v1 to v2\} is regular
	\item L2 = \{words: red count = orange count = yellow count\} is semi-linear
\end{itemize}

\section{PDA example}

\subsection{problem}
money = 0
\begin{itemize}
	\item M is PDA, run M on input w, see pushes and pops on the stack
	\item push a: money += 5
	\item pop b: money -= 10
	\item M is good if there is w meeting condition: money > 0
	\item question: is M good?
\end{itemize}

\subsection{problem is decidable}
\begin{itemize}
	\item L = \{run: run meets condition\}
	\item run is the sequence of push-a(A) and pop-b(B)
	\item L = L1 intersect L2
	\item L2 = \{word: 5 A count - 10 B count > 0\} is semi-linear and commutative
	\item L1 = all possible sequence
	\item construct PDA N to accept L1:
	\begin{enumerate}
		\item run M on w (guess)
		\item the run generates sequence, check the guess is correct
		\item if M accepts w and the entire sequence is checked, N accept sequence
	\end{enumerate}
\end{itemize}

\section{A is in P}
B in P, so we have a deterministic algorithm algB to solve B.
A reduces to B, we have function f
1. f runs polynomial time
2. for all x, x in A iff f(x) in B

we can construct deterministic polynomial time algorithm algA (input = x) to solve A:
return algB(f(x))

algA runs in polynomial time.

\section{in NP complete?}
cracking RSA: unknown
graph isomorphism: unknown

\section{3SAT is NP complete}
translate: x1 or x2 or x3 bar or x4 bar or x5 or x6 bar to (x1 or x2 or y1) and 
(y1 bar or x3 bar or y2) and (y2 bar or x4 bar or y3) and (y3 bar or x5 or x6 
bar)

\section{IDSET is NP complete}
translate (x1 or x2 or x3 bar) and (x1 or x2 bar or x3) and (x1 or x2 bar or 
x3) and (x1 or x2 or x3) to graph of triangles

\section{VC is NP complete}
C is VC of G iff V-C is an IDSET of G

\end{document}