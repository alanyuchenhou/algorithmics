\documentclass{article}
\usepackage{listings}
\usepackage{amsmath}
\usepackage{fullpage}
\usepackage{tabularx}
\usepackage{graphicx}
\usepackage{cite}
\begin{document}
\lstset{language=python, tabsize=4}
\title{CS 516 homework 2}
\author{Yuchen Hou}
\maketitle

\section{TM: change}

\subsection{10-change TM}

The smallest class is Regular, because any 10-change TM can be simulated by a FA. The proof is as follows:
\begin{enumerate}
	\item use FA's read to simulate TM's write;
	\item use a table to keep track of changes;
	\item when TM moves left, FA moves left;
	\item when TM moves right, FA moves right;
	\item for every symbol, use homomorphism to generate the new symbol sequence for the FA, \item taking into account the symbols the TM is going to change;
	\item let the FA work on the new symbol sequence;
\end{enumerate}

\subsection{10-change-per-cell TM}

$ \forall TM M_1 $, we can construct a 10-change-per-cell TM M to simulate $ M_1 $:
\begin{lstlisting}
def M(w):
	run M_1 on w:
		every time M_1 changes a symbol, shift the head and symbols in the
		non-blank region to the blank region on the right
	if M_1 returns yes:
		return yes
\end{lstlisting}
Notice that M changes a cell at most twice.

\section{1-turn TM}

The smallest class is Regular, because any 1-turn TM can be simulated by a FA. The proof is as follows:
\begin{enumerate}
	\item the input of the FA has 4 tracks;
	\item track 1: the 1st symbol the TM reads during the right move;
	\item track 2: the last symbol the TM reads during the right move;
	\item track 3: the 1st symbol the TM reads during the left move;
	\item track 4: the last symbol the TM reads during the left move;
	\item the FA makes only one rightward move and simulate the backward move of the TM along the way;
		
\end{enumerate}

\section{Jump TM}

$ \forall$ TM N, we can construct a jump TM M to simulate N:
\begin{lstlisting}
def M(w):
	run on w like N:
		if N would move left to cell C:
			jump to the left end
			move right a few times
			mark C(guess where it is)
			move right once(verify the marked cell was the right cell)
			jump to the left end
			move right a few times
			stop at C
		else:
			run exactly like N
\end{lstlisting}

\section{Simulating 2 counters with 1 queue}
The queue simulates 2 counters: A and B. The queue alphabet = \{A, B, 1\}.
\begin{lstlisting}

def __init__():
	"""
	initialize 2 counters: A = 0, B = 0
	"""
	queue.enqueue(A)
	queue.enqueue(B)
	
def increment(X):
	"""
	increment counter X
	
	@param: X is the counter to be incremented, either A or B
	"""
	symbol = none
	while (symbol != X):
		symbol = queue.dequeue()
		queue.enqueue(symbol)
	queue.enqueue(1)

def decrement(X):
	"""
	decrement counter X
	
	@param: X is the counter to be decremented, either A or B
	@crash: if X == 0
	"""
	symbol = none
	while (symbol != X):
		symbol = queue.dequeue()
		queue.enqueue(symbol)
	symbol = queue.dequeue()
	assert(symbol == 1)
	
def is_zero(X):
	"""
	test if counter X == 0
	
	@param: X is the counter to be decremented, either A or B
	@return: true if X == 0, false otherwise
	"""
	symbol = none
	while (symbol != X):
		symbol = queue.dequeue()
		queue.enqueue(symbol)
	symbol = queue.dequeue()	
	queue.enqueue(symbol)
	if symbol == 1:
		return false
	else:
		return true
	
\end{lstlisting}	

\section{}

\section{}

The machine can be simulated with a bounded stack machine(finite memory), which can be simulated by a FA. The existence of B is equivalent to the non-emptiness of the FA, which is decidable.

\end{document}