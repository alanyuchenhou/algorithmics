\documentclass{article}
\usepackage{listings}
\usepackage{amsmath}
\usepackage{fullpage}
\usepackage{tabularx}
\usepackage{graphicx}
\usepackage{cite}
\begin{document}
\lstset{language=python, tabsize=4}
\title{CS 516 homework 2}
\author{Yuchen Hou}
\maketitle

\section{TM: change}

\subsection{10-change-TM is FA}
The smallest class is Reg, because given 10-change TM N, there is FA M (input = 
w) to simulate N:
\begin{enumerate}
	\item guess all 10 changes N will make and generate the input after changes
	\item M moves like N except it does not change any symbols
\end{enumerate}
Therefore L(M) is Reg.

\subsection{10-change-per-cell TM is TM}
Given TM N, there is 10-change-per-cell TM M (input = w) to simulate N:
\begin{enumerate}
	\item if N writes a symbol: shift the head and symbols in the non-blank 
	region to the blank region on the right
	\item else: move like N
\end{enumerate}
Notice that M changes a cell at most twice.

\section{1-turn TM is FA}
The smallest class is Reg, because given 1-turn TM M, there is FA N (input = w) 
to simulate M:
\begin{enumerate}
	\item guess all changes M will make and generate the input after changes 
	with 4 tracks:
	\begin{enumerate}		
		\item the 1st symbols M reads during the right move
		\item the last symbols M reads during the right move
		\item the 1st symbols M reads during the left move
		\item the last symbols M reads during the left move
	\end{enumerate}
	\item make pass and simulate the backward move of M along the way
\end{enumerate}

\section{Jump-TM is TM}
Given TM N, there is jump-TM M (input = w) to simulate N:
\begin{enumerate}
	\item if N moves left: move left(jump left, move right a few times, stop(guess when to stop) and mark), move right to verify the marked cell is right, jump left, move right a few times, stop at mark
	\item else: move like N
\end{enumerate}

\section{Simulating 2 counters with 1 queue}
The queue simulates 2 counters: A and B. The queue alphabet = \{A, B, 1\}.
\begin{lstlisting}

def __init__():
	"""
	initialize 2 counters: A = 0, B = 0
	"""
	queue.enqueue(A)
	queue.enqueue(B)
	
def increment(X):
	"""
	increment counter X
	
	@param: X is the counter to be incremented, either A or B
	"""
	symbol = none
	while (symbol != X):
		symbol = queue.dequeue()
		queue.enqueue(symbol)
	queue.enqueue(1)

def decrement(X):
	"""
	decrement counter X
	
	@param: X is the counter to be decremented, either A or B
	@crash: if X == 0
	"""
	symbol = none
	while (symbol != X):
		symbol = queue.dequeue()
		queue.enqueue(symbol)
	symbol = queue.dequeue()
	assert(symbol == 1)
	
def is_zero(X):
	"""
	test if counter X == 0
	
	@param: X is the counter to be decremented, either A or B
	@return: true if X == 0, false otherwise
	"""
	symbol = none
	while (symbol != X):
		symbol = queue.dequeue()
		queue.enqueue(symbol)
	symbol = queue.dequeue()	
	queue.enqueue(symbol)
	if symbol == 1:
		return false
	else:
		return true
	
\end{lstlisting}

\end{document}
