\documentclass{article}
\usepackage{listings}
\usepackage{amsmath}
\usepackage{fullpage}
\usepackage{tabularx}
\usepackage{graphicx}
\usepackage{cite}
\begin{document}
\lstset{language=python}
\title{CS 516 midterm}
\author{Yuchen Hou}
\maketitle

\section{decidability}

\subsection{algorithm design}

\subsubsection{key words}
decidable, recursive, there is an algorithm

\subsubsection{solution}
no efficiency concern, anything that works is fine.

\section{undecidability}

\subsection{key words}
undecidable, no algorithm

\subsection{solution}
halting problem is undecidable of 2CM

\section{undecidability: emptiness problem for TM}
standard undecidability proof

\section{RE proof}
proof $ L \in RE $.

\subsection{solution}
write a function:
def TM(word):
	if $ word \in L $:
		return yes

\section{$ L \in RE $}
$ L' = \{y: ya \in L \} \in RE $

\subsection{Proof}
$ L \in RE \implies \exists$ TM such that:

def TM(word):
	if $ word \in L $:
		return yes

We construct M(word):

def M(word):
	if TM(word + a) == yes:
		return yes
		
\section{$ L_1, L_2 \in R$}
proof $ L = \{xy: x \in L_1, y \in _2\} \in R$
	
There are 2 algorithms $ M_1, M_2 $ to recognize $ L_1, L_2 $.
Construct M to recognize to L:
\begin{itemize}
	\item input: w
	\item cut w into x and y f, enumerate all possibilities
	\item for each pair (x, y) \{if M1(x) == yes and M2(y) == yes return yes\}
	\item return no
\end{itemize}

\section{undecidability}

\subsection{reduction}
Language A and B. Proof: if A is undecidable, B is undecidable.

\subsection{proof by contradiction}

assume B is decidable, there is algorithm M to recognize B. construct algorithm N to recognize A.

\end{document}