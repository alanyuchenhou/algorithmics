\documentclass{article}
\usepackage{listings}
\usepackage{amsmath}
\usepackage{fullpage}
\usepackage{tabularx}
\usepackage{graphicx}
\usepackage{cite}
\begin{document}
\lstset{language=python, tabsize=4}
\title{CS 516 homework 4}
\author{Yuchen Hou}
\maketitle

\section{Termination of Presburger formula is undecidable}
Proof by reduction (from A-2CM to Termination of Presburger formula) and 
contradiction (assumption: Termination of Presburger formula is decidable). 
From the assumption, we can construct decider M (input = F(x, y, z, x', y', 
z'), where F is a formula ) to decide termination of Presburger formula:
\begin{enumerate}
	\item if F terminates: accept F
	\item else: reject F
\end{enumerate}
We can use M to decide A-2CM N:
\begin{enumerate}
	\item encode N's transition function $ \delta(q, x, y) $ 
	(where q is the state and x, y are counts) as formula F(q, x, y, q', x', 
	y') such 
	that F(q, x, y, q', x', y') = true $ \iff \delta(q, x, y) = (q', x', y') $
	\item N terminates $ \iff $ N has an accepting computation history: $ 
	\exists n \exists q_1 ... q_n, x_1 ... x_n, y_1 ... y_n : F(0, 0, 0, 
	q_1, x_1, y_1) \land ... \land F(q_n, x_n, y_n, 1, 1, 1) = true$
\end{enumerate}
This is a contradiction with A-2CM is undecidable. Therefore termination of 
Presburger formula is undecidable.

\section{LP is decidable}
Proof by reduction: from LP to satisfiability of Presburger formula. LP is 
equivalent to given formula $ F(x_1 ... x_n, y) $ whether the following formula 
is satisfiable:
\begin{align*}
	LP \iff& \forall x_1 ... x_n \in Z (K \ge y \land F(x_1 ... x_n)) \land 
	\exists x_1 ... x_n \in Z (K = y \land F(x_1 ... x_n) )
\end{align*}
This is the satisfiability of Presburger formula, a decidable problem. 
Therefore LP is decidable.

\section{A Linear formula is a Presburger formula}

\section{Satisfiability of mixed-linear formula is decidable}

Proof by reduction from satisfiability of mixed-linear formula to linear 
formula or to linear formula and Presburger formula.

\subsection{Reduction to linear formula}
A mixed-linear-formula f can be reduced to a linear formula by boolean variable 
elimination:
\begin{align*}
	f(x_1 ... x_k, y_1 ... y_m)
	&= f(x_1 ... x_k, 0 ... 0)\\
	&\lor f(x_1 ... x_k, 0 ... 1)\\
	&\lor f(x_1 ... x_k, 1 ... 0)\\
	&\lor f(x_1 ... x_k, 1 ... 1)\\
\end{align*}
The additive theory of reals is decidable, therefore satisfiability of 
mixed-linear-constraint is decidable.

\subsection{Reduction to linear formula}
A mixed-linear-formula f can be separated into 2 parts - an integral part and a 
real part - by splitting every real variable into an integral part and a real 
part: $ y = z + w $ where y is real, z is integer and y is real in range(0, 1). 
Then we do quantifier elimination on each part separately. The satisfiability 
of real formula and Presburger formula is decidable, therefore the 
satisfiability of mixed-linear formula is satisfiable.

\section{P is Presburger}
A k-vector-adder M is a FA, where W is the addition of some constant and a 
sequence of input symbols.
\begin{align*}
	P(W, W')
	\iff& \exists q, q' ((q, W) \rightarrow* (q', W'))\\
	\iff& \exists q, q' ((q, 0) \rightarrow* (q', W'-W))\\
	\iff& P(0, W' - W)
\end{align*}
M accepts Regular language L. We can construct semi-linear language $ K = \{w 
\mid \#(w) = W'-W\} $. Semi-linear language $ J = L \cap K $ is defined by P. 
Therefore, P is Presburger.

\section{4 tanks make a 2CM}

\section{Stability of PDA is decidable}
We define $ \Sigma = \{A(push \ a), B(push \ b), C(pop \ a), D(pop \ b)\} $.
We can construct regular language $ L_1 = \Sigma $* and semi-linear language $ 
L_2 = \{w | -2\#_Aw + 3\#_Bw \ge 0 \land 3(-2\#_Aw + 3\#_Bw) \ge \#_Aw-\#_Cw + 
\#_Bw-\#_Dw\} $. We can construct PDA N (input = w) to accept Semi-linear 
language $ L = L_1 \cap L_2 $:
\begin{enumerate}
	\item run M:
	\begin{enumerate}
		\item for each action $ \gamma_1 \in \Sigma $: read $ \gamma_2 $ 
		from w and assert $ \gamma_1 = \gamma_2 $
	\end{enumerate}
	\item assert M is in accepting state
	\item assert w is completely read
	\item accept w
\end{enumerate}
M is stable is equivalent to N accepts non-empty language, which is decidable.

\end{document}