\documentclass{article}
\usepackage{listings}
\usepackage{amsmath}
\usepackage{fullpage}
\usepackage{tabularx}
\usepackage{graphicx}
\usepackage{cite}
\begin{document}
\lstset{language=python, tabsize=4}
\title{CS 516 homework 4}
\author{Yuchen Hou}
\maketitle

\section{Termination of Presburger formula is undecidable}
Proof by reduction (from A-2CM to Termination of Presburger formula) and 
contradiction (assumption: Termination of Presburger formula is decidable). 
From the assumption, we can construct decider M (input = [F(x, y, z, x', y', 
z')], where F is a formula) to decide termination of Presburger formula:
\begin{enumerate}
	\item if F terminates: accept [F]
	\item else: reject [F]
\end{enumerate}
We can construct decider N (input = [O], where O is a 2CM) to decide A-2CM:
\begin{enumerate}
	\item construct formula F(q, x, y, q', x', y') to terminate if O halts:
	\begin{enumerate}
		\item F(q, x, y, q', x', y') = true $ \iff \delta(q, x, y) = (q', x', 
		y') $, where $ \delta(q, x, y) $ is the transition function of O, q is 
		the state and x, y are counts
		\item F terminates $ \iff $ O has an accepting computation history: $ 
		\exists n \exists q_1 ... q_n, x_1 ... x_n, y_1 ... y_n : F(0, 0, 0, 
		q_1, x_1, y_1) \land ... \land F(q_n, x_n, y_n, 1, 1, 1) = true$, where 
		(0, 0, 0) and (1, 1, 1) are the initial and final configurations
	\end{enumerate}
	\item run M on [F]
	\item if M accepts [F]: accept [O]
	\item else: reject [O]
\end{enumerate}
This is a contradiction with A-2CM is undecidable. Therefore termination of 
Presburger formula is undecidable.

\section{LP is decidable}
Proof by reduction: from LP to satisfiability of Presburger formula.
We can construct decider M (input = [n, f], where n is a number, 
f is a linear function) to decide LP:
\begin{enumerate}
	\item reduce $ f(x_1...x_n) = y $ to a Presburger formula $ F(x_1...x_n, y) 
	s.t. $
	\begin{align*}
		\forall x_1...x_n, y \in Z (F(x_1...x_n,y) = true \iff f(x_1...x_n) =y)
	\end{align*}
	\item reduce formula in question to Presburger formula G(K, y):
	\begin{align*}
		G(K, y): \forall& x_1 ... x_n \in Z(K \ge y \land F(x_1 ... x_n, y)) \\
		\land \exists& x_1 ... x_n \in Z (K = y \land F(x_1 ... x_n, y))
	\end{align*}
	\item if G(K, y) is satisfiable: accept [n, f]
	\item else: reject [n, f]
\end{enumerate}
Therefore LP is decidable.

\section{A Linear formula is a Presburger formula}

\section{Satisfiability of mixed-linear formula is decidable}

Proof by reduction from satisfiability of mixed-linear formula to 
satisfiability of linear formula (or linear formula and Presburger formula).

\subsection{Reduction to linear formula}
We can construct decider M (input = [f], where f is a mixed linear formula) to 
decide satisfiability of mixed-linear formula:
\begin{enumerate}
	\item reduce $ f(x_1 ... x_k, y_1 ... y_m) $ to linear formula $ g(x_1 ... 
	x_k) $:
	\begin{align*}
		g(x_1 ... x_k):& f(x_1 ... x_k, 0 ... 0) \\
		 \lor& f(x_1 ... x_k, 0 ... 1) \\
		& ... \\
		\lor& f(x_1 ... x_k, 1 ... 0) \\
		\lor& f(x_1 ... x_k, 1 ... 1)
	\end{align*}
	\item if $ g(x_1 ... x_k) $ is satisfiable: accept [f]
	\item else: reject [f]
\end{enumerate}
Therefore satisfiability of mixed-linear formula is decidable.

\subsection{Reduction to linear formula and Presburger formula}
We can construct decider M (input = [f], where f is a mixed linear formula) to 
decide satisfiability of mixed-linear formula:
\begin{enumerate}
	\item split every real variable in f into an integral part and a real 
	part in range(0, 1)
	\item reduce f to $ g \land h $ where g is a real formula and h is a 
	Presburger formula
	\item if g is satisfiable and f is satisfiable: accept [f]
	\item else: reject [f]
\end{enumerate}
Therefore the satisfiability of mixed-linear formula is decidable.

\section{P is Presburger}
A k-vector-adder M is a FA, where W is the addition of some constant and a 
sequence of input symbols.
\begin{align*}
	P(W, W')
	\iff& \exists q, q' ((q, W) \rightarrow* (q', W'))\\
	\iff& \exists q, q' ((q, 0) \rightarrow* (q', W'-W))\\
	\iff& P(0, W' - W)
\end{align*}
M accepts Regular language L. We can construct semi-linear language $ K = \{w 
\mid \#(w) = W'-W\} $. P defines semi-linear language $ J = L \cap K $. 
Therefore, P is Presburger.

\section{4 tanks make a 2CM}

\section{Stability of PDA is decidable}
Proof by reduction from stability of PDA to E-PDA. We can construct decider O 
(input = [M], where M is a PDA) to decide stability of PDA:
\begin{enumerate}
	\item define $ \Sigma = \{A(push \ a), B(push \ b), C(pop \ a), D(pop \ 
	b)\} $
	\item construct semi-linear language L:
	\begin{align*}
		L = \{w | & -2\#_Aw + 3\#_Bw \ge 0 \\
		\land & 3(-2\#_Aw + 3\#_Bw) \ge \#_Aw-\#_Cw + \#_Bw-\#_Dw\}
	\end{align*}
	\item construct PDA N (input = w) to accept Semi-linear language L:
	\begin{enumerate}
		\item run M, for each action $ \gamma_1 \in \Sigma $: read $ \gamma_2 $ 
		from w and assert $ \gamma_1 = \gamma_2 $
		\item assert M is in accepting state
		\item assert w is completely read
		\item accept w
	\end{enumerate}
	\item if L(N) is empty: reject [M]
	\item else: accept [M]
\end{enumerate}
Therefore the stability of PDA is decidable.

\end{document}