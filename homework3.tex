\documentclass{article}
\usepackage{listings}
\usepackage{amsmath}
\usepackage{fullpage}
\usepackage{tabularx}
\usepackage{graphicx}
\usepackage{cite}
\begin{document}
\lstset{language=python, tabsize=4}
\title{CS 516 homework 3}
\author{Yuchen Hou}
\maketitle

\section{emptiness of 2-way FA with counter is undecidable}
approach 2
tape: head, junk, junk, junk, ... tail
simulate 2CM with target machine:
2CM / target
state transition / state transition
update x / update counter
increment y / move right
decrement y / move left
test y == 0 / test head = state
this simulation is unfaithful in that the target crashes upon head moves to the right of tail
\subsection{observations}
2CM halts $ \iff \exists $ w accepted by target machine 

\section{emptiness of FA with 2 heads is undecidable}
approach 1
2CM halts $ \iff L(M) \neq \emptyset$

\section{2-stack 1-turn PDA}
approach 1 with reverse configuration sequence
target machine reads through the sequence and push every configurations on the 2 stacks
stack 2 pop once
stack 1 and stack 2 pop together and validate the transitions

\section{algorithm design}
a counter is a stack with unary alphabet. replace the stack with counter. make 2 cuts on the counters evolution at their peaks, into monotonic region. check the 4 values at the 2 cuts and analyze the evolution between the 2 peaks with linear equations???? too hard!!!!

\section{Acceptance problem for TM is undecidable}

\section{Emptiness problem for jump-LBA is undecidable}

\section{Language is RE}

\section{Reverse of R is R}

\section{Emptiness problem for read-only 2-TM is undecidable}

\section{2CM 2 counter machine simulating a TM}
Given TM M, is L(M) = empty? is undecidable.
treat every TM as a 2CM. A TM configuration:
config = #state#count1#count2#
execution:
config1#instruction4#config2#instruction87#
2 approaches:
\begin{enumerate}
	\item feed the arbitrary execution of the 2-counter machine to the target machine
	\item simulate the 2-counter machine with the target machine
\end{enumerate}

\end{document}
