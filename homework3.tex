\documentclass{article}
\usepackage{listings}
\usepackage{amsmath}
\usepackage{fullpage}
\usepackage{tabularx}
\usepackage{graphicx}
\usepackage{cite}
\begin{document}
\lstset{language=python, tabsize=4}
\title{CS 516 homework 3}
\author{Yuchen Hou}
\maketitle

\section{Emptiness problem for 2-way FA with counter is undecidable}

\subsection{Observations and approach}
We can reduce halting problem of 2CM(undecidable) to emptiness problem of 2-way FA with 1 counter(2-1 FA for short) to show it is undecidable.

\subsection{Proof}
We can construct a 2-1 FA M to simulate the 2CM N this way:
\begin{itemize}
	\item create this tape for M to work on: [begin, ?, ?, ... ?, end], where begin and end are special tape symbols indication the begin and end of the tape; ? is a any tape symbol;
	\item let 2 counters of N be c1 and c2; let the counter of M be c3;
	\item use c3 to simulate c2;
	\item M has the same states as N;
	\item M simulates N following the rules in this table:
		
	\begin{tabularx}{0.5\textwidth}{|X|c|}  \hline
		 N's action & M's action \\ \hline
		 state transition & same state transition \\ \hline
		 operation on c2 & same operation on c3 \\ \hline
		 increment c1 & move head right \\ \hline
		 decrement c1 & move head left \\ \hline
		 test c1 == 0 & test head pointing to begin \\ \hline
	\end{tabularx}
\end{itemize}
This simulation is unfaithful because M crashes upon head moves to the right of tail. However, we still have the conclusion that N halts $ \iff \exists w \in L(M) \iff L(M) \neq \emptyset$. Therefore, the emptiness problem for 2-way FA with counter is undecidable.

\section{Emptiness problem for FA with 2 heads is undecidable}

\subsection{Observations and approach}
We can reduce halting problem of a 2CM(undecidable) N to emptiness problem of a 2-head FA M to show it is undecidable.

\subsection{Proof}
We can run M on the execution of N as follows:
\begin{itemize}
	\item record the execution of N on the tape in this format: [configuration[1], configuration[2] ... configuration[n]], where configuration[i] = (instruction, state, counter1, counter2) is the ith configuration in the execution;
	\item move head2 to the 2nd configuration;
	\item let 2 heads move forward at the same speed while head1 stays 1 configuration behind head2, so that they can check the each transition from one configuration to the next one is valid;
	\item verify the 1st configuration is valid initial configuration;
	\item for each transition:
	\begin{itemize}
		\item head1 reads the instruction; M changes state to remember the instruction;
		\item 2 heads read the 2 states and verify the state transition is valid; M changes state to remember the 2 states;
		\item 2 heads read the 2 values for counter1 and verify the its update is valid;
		\item 2 heads read the 2 values for counter2 and verify the its update is valid;
	\end{itemize}
	\item verify the last configuration is valid final configuration;
\end{itemize}
N halts $ \iff \exists w \in L(M) \iff L(M) \neq \emptyset$. Therefore, the emptiness problem for FA with 2 heads is undecidable.

\section{Emptiness problem for 2-stack 1-turn PDA is undecidable}

\subsection{Observations and approach}
We can reduce halting problem of a 2CM(undecidable) N to emptiness problem of a 2-stack 1-turn M to show it is undecidable.

\subsection{Proof}
We can run M on the execution of N as follows:
\begin{itemize}
	\item record the execution of N on the tape in the reverse of this format: [configuration[1], configuration[2] ... configuration[n]], where configuration[i] = (instruction, state, counter1, counter2) is the ith configuration in the execution;
	\item M reads through the tape and pushs every symbol on both stacks
	\item stack 2 pops 1 configuration
	\item let 2 stack pop at the same speed while stack1 keeps 1 configuration more than stack2, so that they can check the each transition from one configuration to the next one is valid;
	\item verify the 1st configuration is valid initial configuration;
	\item for each transition:
	\begin{itemize}
		\item stack1 pops and reads the instruction; M changes state to remember the instruction;
		\item 2 stacks pop and read the 2 states and verify the state transition is valid; M changes state to remember the 2 states;
		\item 2 stacks pop and read the 2 values for counter1 and verify the its update is valid;
		\item 2 stacks pop and read the 2 values for counter2 and verify the its update is valid;
	\end{itemize}
	\item verify the last configuration is valid final configuration;
\end{itemize}
N halts $ \iff \exists w \in L(M) \iff L(M) \neq \emptyset$. Therefore, the emptiness problem for 2-stack 1-turn PDA is undecidable.

\section{algorithm design}
a counter is a stack with unary alphabet. replace the stack with counter. make 2 cuts on the counters evolution at their peaks, into monotonic region. check the 4 values at the 2 cuts and analyze the evolution between the 2 peaks with linear equations???? too hard!!!!

\section{Acceptance problem for TM is undecidable}

\section{Emptiness problem for jump-LBA is undecidable}

\section{Language is RE}

\section{Reverse of R is R}

\section{Emptiness problem for read-only 2-TM is undecidable}

\section{2CM 2 counter machine simulating a TM}
Given TM M, is L(M) = empty? is undecidable.
treat every TM as a 2CM. A TM configuration:
config = state;count1;count2;instruction4;
execution:
config1;config2
2 approaches:
\begin{enumerate}
	\item feed the arbitrary execution of the 2-counter machine to the target machine
	\item simulate the 2-counter machine with the target machine
\end{enumerate}

\end{document}
