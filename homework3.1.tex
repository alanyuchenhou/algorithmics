\documentclass{article}
\usepackage{listings}
\usepackage{amsmath}
\usepackage{fullpage}
\usepackage{tabularx}
\usepackage{graphicx}
\usepackage{cite}
\begin{document}
\lstset{language=python, tabsize=4}
\title{CS 516 homework 3.1}
\author{Yuchen Hou}
\maketitle

\section{E-2-way+1-counter-FA is undecidable}
Proof by reduction(from A-2CM to E-2-way+1-counter-FA) and contradiction 
(assumption: E-2-way+1-counter-FA is decidable).
We can construct decider O(input = [P], where P is a 2CM) to decide A-2CM:
\begin{enumerate}
	\item construct 2-way+1-counter-FA Q(input = [w], where w is a word) to 
	accept non-empty language if P halts:
	\begin{enumerate}
		\item simulate P with its counter simulating the 1st counter in P and its head position on the tape simulating the 2nd counter in P
		\item if the simulation finishes: accept w
	\end{enumerate}
	\item if L(Q) is empty (decidable by assumption): reject [P]
	\item else: accept [P]
\end{enumerate}
This is a contradiction with A-2CM is undecidable. Therefore E-2-way+1-counter-FA is undecidable.

\section{E-2-head-FA is undecidable}
Proof by reduction(from A-2CM to E-2-head-FA) and contradiction (assumption: 
E-2-head-FA is decidable).
We can construct decider O(input = [P], where P is a 2CM) to decide A-2CM:
\begin{enumerate}
	\item construct 2-head-FA Q(input = [w], where w is a word) to accept accepting computation histories of P:
	\begin{enumerate}
		\item assert w has valid computation history format: [c[1], c[2], ..., c[l]], where c[i] = [instruction, state, counter1, counter2]
		\item assert c[1] is the initial configuration for P
		\item assert all configuration transitions are valid (2 heads check adjacent configurations: read instruction, check state and counter updates)
		\item assert c[l] is an accepting configuration
		\item accept w
	\end{enumerate}
	\item if L(Q) is empty (decidable by assumption): reject [P]
	\item else: accept [P]
\end{enumerate}
This is a contradiction with A-2CM is undecidable. Therefore E-2-head-FA is undecidable.

\section{E-2-stack-push-pop-PDA is undecidable}
Proof by reduction(from A-2CM to E-2-stack-push-pop-PDA) and 
contradiction(assumption: E-2-stack-push-pop-PDA is decidable).
We can construct a decider O(input = [P], where P is a 2CM) to decide A-2CM:
\begin{enumerate}
	\item construct a 2-stack-push-pop-PDA Q(input = [w], where w is a word) to 
	accept accepting computation histories of P:
	\begin{enumerate}
		\item read every symbol in w, push it on both stacks, and assert w is the reverse of a valid computation history format: [c[1], c[2], ..., c[l]], where c[i] = [instruction, state, counter1, counter2]
		\item assert c[1] is the initial configuration for P (1 stack pops c[1])
		\item assert all configuration transitions are valid (2 stacks pop and check adjacent configurations: read instruction, check state and counter updates)
		\item assert c[l] is an accepting configuration
		\item accept w
	\end{enumerate}
	\item if L(Q) is empty (decidable by assumption): reject [P]
	\item else: accept [P]
\end{enumerate}
This is a contradiction with A-2CM is undecidable. Therefore E-2-stack-push-pop-PDA is undecidable.

\section{E-2-counter-increment-decrement-FA is decidable}
make 2 cuts on the counters evolution at their peaks, to cut it into 3 monotonic regions. check the 4 values at the 2 cuts and analyze the evolution between the 2 peaks with linear equations.

\end{document}
